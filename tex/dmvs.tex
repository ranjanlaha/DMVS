\documentclass[aps,prl,10pt,twocolumn,superscriptaddress,showpacs]{revtex4-1}
\usepackage{titlesec}
\usepackage{hyperref}
\usepackage{graphicx}
\usepackage{amsfonts,amsmath,amssymb,bm,bbm}
\usepackage{color}

\hypersetup{
    pdfnewwindow=true,      % links in new window
    colorlinks=true,       % false: boxed links; true: colored links
    linkcolor=blue,          % color of internal links
    citecolor=blue,        % color of links to bibliography
    filecolor=blue,      % color of file links
    urlcolor=blue        % color of external links
}

\titleformat{\section}[runin]
  {\normalfont\bfseries}{\thesection}{1em}{}[:]
\titlespacing*{\section}{0cm}{2em}{1em}
\titleformat{\subsection}[runin]
  {\normalfont\itshape}{\thesubsection}{1em}{}[:]
\titlespacing*{\subsection}{1cm}{2em}{1em}

\begin{document}

\title{The Doppler effect on indirect detection of decaying dark matter}
\author{Devon Powell}
\email{dmpowel1@stanford.edu}
\author{Ranjan Laha}
\email{rlaha@stanford.edu}
\affiliation{Kavli Institute for Particle Astrophysics and Cosmology (KIPAC),
	\\ Department of Physics, Stanford University, Stanford, CA 94035, USA\\
		SLAC National Accelerator Laboratory, Menlo Park, CA 94025, USA}
\date{\today}

\begin{abstract}

\end{abstract}

%\pacs{95.35.+d, 13.35.Hb, 14.60.St, 14.60.Pq}
% 95.35.+d Dark matter
% 13.35.Hb Decays of heavy neutrinos
% 14.60.St Non-standard-model neutrinos, right-handed neutrinos, etc.
% 14.60.Pq Neutrino mass and mixing 

\maketitle


% % % % % % % % % % % INTRODUCTION % % % % % % % % % % % % % % % % % % % % % %
\section{Introduction}
\label{sec:Introduction}

For a large field of view instrument\,\cite{Figueroa-Feliciano:2015gwa}:
\begin{eqnarray}
\mathcal{F} = \dfrac{\Gamma}{4\pi \, m_s} \, \int_{\Omega} \int _0 ^\infty d\Omega \, ds \, \rho[r(s, \Omega)] \, .
\label{eq:flux for large FoV}
\end{eqnarray}
We can rewrite Eqn.\,\ref{eq:flux for large FoV} as 
\begin{eqnarray}
\dfrac{d^2 \mathcal{F}}{d\Omega \, dE} =  \dfrac{\Gamma}{4\pi \, m_s} \, \int _0 ^\infty  \, ds \, \rho[r(s, \Omega)] \, \dfrac{dN(E)}{dE} .
\label{eq:double differential for the flux}
\end{eqnarray}

Similar to the previous paper, we can write
\begin{eqnarray}
\dfrac{d \tilde{N} (E, r[s, \Omega])}{dE} =\int dE' \, \dfrac{dN(E')}{dE'} \, G(E - E', \sigma_{E'}) \, ,
\label{eq:formula for modified dNdE}
\end{eqnarray}
where the convolution function $G(E, \sigma_E)$ takes the form of a Gaussian with an width of $\sigma_E = (E/c) \sigma_{v_{\rm LOS}}$.  We assume that $\sigma_{v_{\rm LOS}} \approx \sigma_{v_r}(r[s,\Omega])$.

I will now show the derivation of these formulae.  Let us assume that the velocity distribution is $f(v)$ and the differential spectrum is $dN/dE = \delta (E- E_0)$.  The effect of including this velocity distribution is that it takes the mono energetic spectrum to $\dfrac{d\tilde{N}}{dE} = \delta \left(E - E_0 (1 \pm \dfrac{v_0}{c})\right)$.  From this we can intuitively derive the following formula which is valid for a general $f(v)$:
\begin{eqnarray}
\dfrac{d\tilde{N}}{dE} = \int f(v) \, \dfrac{dN}{dE'} \, G(E, E') \, dv \, dE' \,
\label{eq:step 1}
\end{eqnarray}
where $G(E, E')$ is the convolution function.  To estimate a functional form of $G(E, E')$, we can use the test case $f(v) = \delta (v - v_0)$, and $dN/dE = \delta (E' - E_0)$ to determine $G(E, E') = \delta (E - E' (1 \pm v/c))$.

Let us now consider $f(v) = \dfrac{1}{\sqrt{2\pi} \sigma_v} \, e^{v^2/2 \, \sigma_v^2}$, so that
\begin{eqnarray}
\dfrac{d\tilde{N}}{dE} &=& \int \delta(E' - E_0) \, \dfrac{1}{\sqrt{2\pi} \sigma} \, e^{v^2/2 \, \sigma^2} \, \nonumber\\
&\times& \delta (E - E' (1 \pm v/c)) \, dv \, dE'.
\label{eq:step2}
\end{eqnarray}
We have $\delta (E - E' (1 \pm v/c)) = \dfrac{c}{E'} \, \delta \left(v + c - \dfrac{E}{E'} c \right)$\, .  We can do the integrals to find
\begin{eqnarray}
\dfrac{d\tilde{N}}{dE} =  \dfrac{1}{\sqrt{2\pi}} \dfrac{c}{\sigma_v E_0} {\rm exp} \left( \dfrac{-(E-E_0)^2}{2 E_0^2 \, \sigma_v^2/c^2} \right)
\label{eq:step3}
\end{eqnarray}
which we can compare with a regular Gaussian to derive $\sigma_E = (E/c) \sigma_v$.


% % % % % % % % % % % METHODS % % % % % % % % % % % % % % % % % % % % % %
\section{Methods}

We use a suite of Milky Way zoom-in simulations run by \cite{mao2015} using the L-GADGET cosmology code
(a descendant of GADGET-2, \cite{springel2005}) to study the Doppler-shifted line emission due to sterile neutrino decay. 

The decay signal spectral intensity is traditionally defined in terms of a line integral along the viewing direction $\psi$:
\begin{equation}
\frac{dI(\psi, E)}{dE} = \frac{\Gamma}{4 \pi m_\chi} \frac{dN(E)}{dE} \int_\phi \rho_\chi(r[s,\phi]) \, ds
\end{equation}
where the integral term is the well-known ``J-factor,'' which captures the enhancement of the 
signal due to substructure.

The main insight of \cite{speckhard2016} is that for a detector with sufficient spectral
resolution, the decay spectrum $\frac{dN(E)}{dE}$ can no longer be considered to be independent of
position.  This is due to Doppler shifting induced by the Sun's motion around the galactic center
as well as broadening due to the position-dependent velocity dispersion $\sigma(r)$.
Qualitatively speaking, the observed spectrum is given by the rest-frame decay spectrum
$\frac{dN(E)}{dE}$ broadened by the dark matter velocity dispersion $\sigma(r)$, shifted by the Sun's velocity 
relative to the dark halo $\delta E_{MW} = \frac{E v_{MW}}{c}$, and integrated along the line of sight (LOS):

\begin{equation} \label{eq:analytic}
\frac{dI(\psi, E)}{dE} = \frac{\Gamma}{4 \pi m_\chi} \int_\phi \rho_\chi(r[s,\phi]) \,
\frac{d\widetilde{N}(E-\delta E_{MW}, r[s,\phi])}{dE} \, ds
\end{equation}

Here, $\frac{d\widetilde{N}}{dE}$ is the rest-frame spectrum broadened by the local
(position-dependent) velocity dispersion. \cite{speckhard2016} model this as a Gaussian convolution
with a width dependent on an analytic prescription for $\sigma(r)$.

Rather than attempting to analytically integrate \eqref{eq:analytic}, we construct the full
spectral intensity seen by the detector directly from the N-body particles, incorporating Doppler shift
and velocity dispersion in a straightforward and natural way.   This is similar in spirit to both the ``sightline'' method employed by 
\cite{lovell2015} and the velocity distribution function sampling of \cite{mao2013}, both of whom
eschew analytic prescriptions in favor of operating directly on the information available in the
simulation data. 

We do this by approximating the LOS integral using a thin cone, then sampling all simulation
particles lying inside the cone.  This sampling cone subtends the solid angle $\Omega_s$ and is
understood to lie along $\psi$, the viewing direction as before.  Replacing the integral with a sum over all particles
$p$ in the sampling cone, we obtain the following expression:

\begin{equation} \label{eq:discrete}
	\frac{dI(\psi, E)}{dE} = \frac{\Gamma}{4 \pi m_\chi}\, \sum_{p \, \in \, \Omega_s}
	\, \frac{1}{r_p^{2}} \, \frac{dN[E(1-v_p/c)]}{dE}
\end{equation}

where $r_p$ is the scalar distance to particle $p$ and $v_p$ is the velocity projected along the
line of sight. Intuitively, we are ``stacking the spectra'' from the individual simulation
particles, with weights reflecting the $r^{-2}$ dependence in the observed flux. One can see that by
considering the LOS velocity of each particle independently, we automatically capture the spectral
convolution introduced by the bulk velocity dispersion. In the special case where
$\frac{dN(E)}{dE}$ is a line, computing the observed spectrum is then as simple as
building a $r^{-2}$-weighted histogram of the LOS velocities for all particles in the sampling cone.


\section{Counting statistics and differentiating line centroids}

Given a telescope with effective area $A_{eff}$ and field of view (FOV) $\Omega_{FOV}$, the number
of photons entering the detector during an exposure of length $t$ is

\begin{equation} 
	N_\gamma = \frac{\Gamma t A_{eff}}{4\pi}
	\, \frac{m_s}{m_\chi} \, \frac{\Omega_{FOV}}{\Omega_s} \, \sum_{p \, \in \, \Omega_s} \, \frac{1}{r_p^{2}} 
\end{equation}


%Again, this photon count is modeled entirely using the N-body data and 




% % % % % % % % % % % CONCLUSIONS % % % % % % % % % % % % % % % % % % % % % %
\section{Conclusions}
\label{sec:conclusions}

 
\vspace{-0.5 cm}
	
% % % % % % % % % % % % % % % % % % % % % % % % % % % % % % % % % % % % % %		

% % % % % % % % % % % % % % % %Acknowledgments % % % % % % % % % % % % % % %
\section*{Acknowledgments} 

Mark Lovell, Yao-Yuan Mao, Chris Davis.

% % % % % % % % % % % % % % % % % % % % % % % % % % % % % % % % % % % % % % % % %	

\newcommand{\mnras}[0]{M.N.R.A.S.}
\bibliographystyle{kp}
\bibliography{references}	

\end{document}
